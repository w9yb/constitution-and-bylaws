\documentclass{article}

\usepackage[hidelinks]{hyperref}
\usepackage{microtype}

\title{\vspace*{\fill}Purdue Amateur Radio Club By-laws}
\author{Prepared By: The Officers of Purdue Amateur Radio Club}
\date{February 22, 2025\vspace*{\fill}\vspace*{\fill}}

\begin{document}

\maketitle
\newpage

\tableofcontents
\newpage

\section{Station}

\subsection{Location}

The club station shall be located in the west tower of the Purdue Memorial
Union.

\subsection{Admission}

Independent ability to unlock the station door by those entitled to access will
be allowed whenever possible through a system maintained by the officers with
any necessary approvals from Purdue Memorial Union building management. Full
members and community members will be allowed access unless their access is
revoked by unanimous vote of club officers. The trustee, the advisor, and anyone
authorized by the Purdue Memorial Union to enter the station space will always
have access. No other people shall be allowed unsupervised access.

Anyone with access to any codes needed to gain access to the station shall guard
them appropriately to ensure they are not disclosed. Codes will be changed if
needed when someone loses access privileges (via revocation or expiration of
membership) to effectuate the change and at any other time the officers consider
it necessary.

If the officers revoke a member's access they can reinstate it at any time by
consensus. Access can also be reinstated by a vote.

People who do not have independent access to the station are permitted to visit
the station as long as someone who has independent access is present and willing
to supervise them.

\subsection{Operation}

All club members are charged with ensuring club property is respected and radio
equipment in club spaces is not used contrary to Federal Communications
Commission regulations or good amateur practice. Operation of the station must
at no time reflect in a detrimental manner on Purdue University. Proper safety
precautions shall be taken at all times in club spaces.

Other rules regarding station operation, conduct within station premises, and
use of club property or station equipment can be adopted by vote.

\subsection{Club Property}

Club property will not be removed from the station premises without approval
through a mechanism adopted by the club. Club equipment must be returned unless
another disposition is approved by a two-thirds vote.

\subsection{Personal Property}

People who bring personal property into the club space accept all responsibility
for loss, damage, and theft. They also must follow all rules adopted by the club
for personal property. In the event removal is provided for by the rules, at
least one week will be given to remove the property, unless it may pose a
hazard.

\section{Regular Meetings}

Regular meetings are held once weekly during the fall and spring semesters. The
day of the week and time of the meetings shall be promulgated by the officers at
the beginning of each semester. The meeting will not occur if classes are not
held on a day.

\section{Committees and Leadership Positions}

The officers---by consensus---may appoint members to positions of leadership to
execute delegated tasks. The term of appointments shall not extend beyond the
end of the current annual officer term. By consensus the officers can remove
appointments. Appointment does not grant officer privileges and all appointed
positions report to the officers.

\section{Budget}

The budgetary year for the club shall be the annual officer term, with the
exception of the first budgetary year after these by-laws are adopted, which
shall begin on January 1, 2025 and last until the end of the annual officer
term. The club budget comprises a number of separate funds. The officer
discretionary fund shall be three-quarters of the dues received during the
current budgetary year minus any money spent from the fund. The general fund
shall be four-fifths of any money not from dues or grants received during the
current budgetary year minus any money spent from the fund. The reserve fund
shall be all other available club funds not received from grants. Grant money
applied for and received shall be managed in its own fund for each grant; each
grant fund will be the money from the relevant grant minus money spent from it.

Money is considered received when it is turned over to the club, not when it is
pledged to the club. Money shall be considered spent and unavailable for future
expenditures as soon as its spending is approved unless the approval is
retracted by vote and the money has not been spent. Money that will be used for
upcoming reimbursements cannot be re-spent.

Any account or transaction fees charged without specific authorization will be
automatically deducted from the officer discretionary fund. If this fund is not
sufficient to cover the charge, the remainder will be deducted from the general
fund, and if this is not sufficient the remainder will be from the reserve fund.

All club spending shall be done in compliance with all Purdue University rules
for student organizations as well as any restrictions placed on the money by the
donor for grants and donations.

\subsection{Spending Proposals}

To be considered, the effect of spending proposals on club funds must be
documented. This can be done by including a section in the proposal from the
treasurer detailing the effect of the proposal, as long as it is current at the
time of voting on the proposal. Alternatively, if the treasurer is in attendance
when the proposal is introduced and voted on, the treasurer can report the
effect on the balance of club funds at such times. The treasurer shall prepare
this section or make this report for any spending proposal submitted to them or
considered during a meeting they are present at. Proposals cannot be considered
if they spend money not currently available in the funds they propose spending
from.

Spending proposals may change which fund a past expense from the same budgetary
year is spent from, as long as it is valid for the proposal to spend money from
that fund.

\subsection{Officer Discretionary Fund}

Money from the officer discretionary fund can be spent for the benefit of the
club by consensus of the officers as they see fit, without approval from the
club membership.

\subsection{General Fund}

Money from the general fund can be spent by vote during a meeting or by approval
of more than half of the entire body of voting members. Proposals to spend money
from the general fund can be introduced by any member during a meeting. Outside
of meetings, by consensus of the officers, a proposal can be introduced to
voting members to seek approval by more than half of them.

\subsection{Reserve Fund}

Money from the reserve fund can only be spent as approved by two-thirds vote.
Proposals to spend money from the reserve fund can only be introduced by at
least half of officers during a meeting. Other members do not have standing to
propose spending from the reserve fund, and should instead submit their proposal
to the officers.

\subsection{Grant Funds}

Each grant applied for and received has its own fund. Spending from these funds
shall be approved in the same manner as the general fund. If a grant proposal
prepared by the officers will designate received money for a specific purpose,
it is tantamount to spending the money for that purpose, so such grant proposals
must be approved in the same way as grant fund spending before being submitted.

\end{document}
