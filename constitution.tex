\documentclass{article}

\usepackage{hyperref}
\usepackage{microtype}

\title{Purdue Amateur Radio Club Constitution}
\author{Prepared By: The Officers of Purdue Amateur Radio Club}
\date{December 6, 2024}
% \vspace*{\fill}

% TODO:
% add provisions about what rules and things can be voted in
% add provisions about what duties not specified can be done by the officers without a vote
% clarify when funds are received for something like PDoG

\begin{document}

\maketitle \vspace{3ex}

\tableofcontents
\newpage

\setcounter{section}{-1}
\section{Preamble}

We, the members of Purdue Amateur Radio Club, declare this constitution the
legal document of Purdue Amateur Radio Club. Members of the club are expected to
subscribe to the following articles. This document shall become effective on
December 13, 2024 and from that time on will replace the previous constitution
and all other documents of this nature.

\section{Name}

The official name of this organization is Purdue Amateur Radio Club. The club is
also known by its Federal-Communications-Commission-assigned callsign W9YB.

\section{Purpose}

The purpose of this club is to serve as a means of maintaining an amateur radio
station for the benefit of its members, to promote goodwill among Purdue amateur
radio operators, to serve as a means of social contact for its members, and to
extend the knowledge and art of amateur radio.

\section{Membership}

\subsection{Eligibility}

Membership and participation are free from discrimination on the basis of race,
religion, color, sex, age, national origin or ancestry, genetic information,
marital status, parental status, sexual orientation, gender identity or
expression, disability, or status as a veteran. All people are eligible for
membership of the club.

Pursuant to the Purdue ``Policy Against Hazing,'' Purdue Amateur Radio Club does
not engage in hazing activities of any kind.

\subsection{Requests}

Membership requests must be made during a regular meeting. The treasurer of the
club shall grant all requests for membership if the requisite dues have been
received unless an objection to the person becoming a member has been made by at
least two members. If such an objection is made, the matter of accepting the
membership request will be put to vote at the next opportunity.

\subsection{Dues}

Membership dues for each available membership duration shall be prescribed by
vote during elections at the end of the prior academic year. At this time, the
club may also adopt a list of people exempt from dues for the following year.
People may also be added to the list by vote.

If the fee for a two semester membership is increased at the end of an academic
year, members who purchased a two-semester membership during the spring term
will owe dues equal to one half of the increase in order to be considered
current once the third regular meeting of the fall semester begins.

\subsection{Duration}

Membership terms shall be the current semester or the current and next
semesters. The term of membership continues until the third regular meeting of
the semester after the last semester of the membership, in order to provide a
period to renew.

\subsection{Types}

\subsubsection{Members}

A member is any person whose term of membership from their most recent
application has not expired and who is current in any payments due to the club.

\subsubsection{Full Members}

Full members are members who are currently Purdue University students
(undergraduate or graduate) at any Purdue University campus and currently hold a
Federal Communications Commission Amateur Radio Operator's License.

\subsubsection{Community Members}

Community members are members who are not Purdue University students, but hold a
valid Federal Communications Commission Amateur Radio Operator's License.

\subsubsection{Associate Members}

Associate members are members who are not full members or community members.

\subsection{Removal}

Members can be removed with approval from at least two-thirds of the entire body
of voting members.

\section{Station}

\subsection{Location}

The club station shall be located in the west tower of the Purdue Memorial
Union.

\subsection{Admission}

Independent ability to unlock the station door by those entitled to access will
be allowed whenever possible through a system maintained by the officers with
any necessary approvals from Purdue Memorial Union building management. Full
members and community members will be allowed access, unless their access is
revoked by unanimous vote of club officers. The trustee, the advisor, and anyone
authorized by the Purdue Memorial Union to enter the station space will always
have access. No other people shall be allowed unsupervised access.

Anyone with access to any codes needed to gain access to the station shall guard
them appropriately to ensure they are not disclosed. Codes will be changed if
needed when someone loses access privileges (via revocation or expiration of
membership) to effectuate the change and at any other time the officers consider
it necessary.

If the officers revoke a member's access they can reinstate it at any time by
consensus. Access can also be reinstated by a vote.

People who do not have independent access to the station are permitted to visit
the station as long as someone who has independent access is present and willing
to supervise them.

\subsection{Operation}

All club members are charged with ensuring club property is respected and radio
equipment in club spaces is not used contrary to Federal Communications
Commission regulations or good amateur practice. Operation of the station must
at no time reflect in a detrimental manner on Purdue University.

Other rules regarding station operation, conduct within station premises, and
use of club property and station equipment can be adopted by vote.

\subsection{Club Property}

Club property will not be removed from the station premises without approval
through a mechanism adopted by the club. Club equipment must be returned unless
another disposition is approved by a two-thirds vote.

\subsection{Personal Property}

Club members who bring personal property into the club space accept all
responsibility for loss, damage, and theft. They also must follow all rules
adopted by the club for personal property. In the event removal is provided for
by the rules, one week will be given to remove the property, unless it may pose
a hazard.

\section{Meetings}

Meetings consist of regular meetings as well as other meetings.

\subsection{Regular Meetings}

Regular meetings are held weekly during the fall and spring semesters except on
days when classes are not in session. The day of the week and time of the
meetings shall be promulgated by the officers at the beginning of each semester.
Official club announcements, such as for elections, and club business requiring
a vote can only be conducted when a quorum is present.

\subsection{Other Meetings}

Other meetings can be called to order anytime a quorum is present to handle
business needing attention. These meetings can occur in person or digitally.

\subsection{Rules}

Rules of order and other rules about how meetings are conducted can be adopted
by two-thirds vote.

\subsection{Voting Members}

Voting members are full members who have been present at one of the prior four
regular meetings, or the current meeting if one is in session.

\subsection{Quorum}

A quorum shall be at least two-thirds of voting members and at least one
officer. Unless otherwise specified, votes will pass if more than half of
members present vote in favor (exactly half will not pass). When the fraction is
specified, only the exact fraction is needed to pass.

\section{Officers and Functionaries}

The club officers shall consist of the president, vice president, treasurer, and
secretary. Officers must be full members, and a person can only hold one officer
role.

The club functionaries are the advisor and the trustee. The advisor must be an
employee of Purdue University, they must not be a student of Purdue University,
and they must meet any other requirements under Purdue University student
organization rules for a club advisor. The trustee must meet all applicable
Federal Communications Commission rules for a club license trustee. One person
can hold both roles.

If an officer or functionary stops meeting their eligibility criteria their
position will be immediately vacated.

The president, treasurer, and functionaries are mandatory positions. Officers
and functionaries can vacate their position either immediately or when an
election to replace them is complete, at their option, by notifying the
officers.

\subsection{Miscellaneous Duties}

This section specified duties of officers in addition to duties assigned to them
by other parts of this document.

\subsubsection{President}

The president is responsible for the re-registration of the club with Purdue
University once per year and presiding over meetings. The president will perform
the duties of the secretary if the position is vacant.

\subsubsection{Vice President}

The vice president will preside in the absence of the president.

\subsubsection{Treasurer}

The treasurer is responsible for maintaining the budget, reporting club fund
balances to full members at meetings if requested, maintaining the roster of
club members, depositing organization funds, and making reimbursements or
expenditures as allowed under this document in a manner approved by the Business
Office for Student Organizations. The treasurer will preside in the absence of
the president and vice president.

\subsubsection{Secretary}

The secretary will keep meeting minutes, do club publicity, publicize elections,
and announce meetings as needed. The secretary will preside in the absence of
the president, vice president, and treasurer.

\subsubsection{Advisor}

The advisor is responsible for signing necessary Purdue University forms for
student organizations and overseeing that operation of the club proceeds in a
manner that complies with Purdue University rules.

\subsubsection{Trustee}

The trustee is responsible for liaising with the Federal Communications
Commission including to maintain the club license(s) and is responsible for the
operation of the station and its equipment. The station license shall be in the
name of the trustee.

\subsection{Elections}

The election process for all positions will begin in the spring semester, at a
time selected by the officers. The time shall be selected so that the elections
will be complete by the fifteenth of April if a quorum exists at every regular
meeting from the start of the process until the election. The new elected
officers will begin their term at the first regular meeting in May, or the last
day of the spring semester, whichever is earlier, but not before the election is
complete.

The election process shall begin immediately if any mandatory positions are
vacant. The election process for other vacant positions may be started at any
time by unanimous vote of the officers. Officers elected to vacant positions in
these special election shall begin their term immediately.

\subsubsection{Election Process}

When the election process begins, it will be announced at the each regular
meeting until the election occurs. Once the announcement is made at a meeting
where a quorum exists, the next regular meeting where a quorum exists shall be
the election.

Members may nominate themselves or others for any position(s) being elected, but
people who object to being nominated shall not be included. Nominations and
objections can be communicated to any of the officers. Nominations are accepted
from the beginning of the election process until after the last call for
nominations at the beginning of the meeting that is the election. All people
nominated for a position that do not object and that are eligible shall be
included on each ballot.

Ballots will be private and ranked choice, but can be in any form approved by
all officers. Elections for functionaries will occur before officers. Elections
will be decided by the ranked pairs method, also known as the Tideman method.
Ties will be broken by the president.

The order of elections for officers shall be president, treasurer, secretary,
vice president. For the elections of secretary and vice president, an option
``none'' will be included on the ballot, and if this option is elected the
positions shall be left vacant. If an officer position is left vacant after any
election, any remaining officer positions will be left vacant and the elections
will end. Once a candidate has been selected to an officer position they will be
eliminated from the ballots of the remaining officer elections.

\subsection{Removal}

Officers or functionaries can be removed, or the election process can be
initiated to replace them, with approval from at least two-thirds of the entire
body of voting members.

\section{Budget}

The budgetary year for the club shall be the annual officer term beginning and
ending each spring. The club budget shall consist of a number of separate funds.
The officer discretionary fund shall consist of three quarters of the dues paid
during the current budgetary year minus any money spent from the fund. The
general fund will consist of four fifths of any money not from dues or grants
received during the current budgetary year minus any money spent from the fund.
The reserve fund shall consist of all other available club funds not received
from grants. Grant money applied for and received shall be managed in its own
fund for each grant; each fund will consist of the money from the relevant grant
minus money spent from it.

Money shall be considered spent and unavailable for future expenditures as soon
as its spending is approved unless the approval is retracted by vote and the
money was unspent. Money that will be used for upcoming reimbursements cannot be
re-spent.

All club spending shall be done in compliance with all Purdue University rules
for student organizations as well as any restrictions placed on the money by the
donor for grants and donations.

\subsection{Spending Proposals}

To be considered, spending proposals must either include a section by the
treasurer detailing the effect of the proposal on the balance of club funds or
if made during a meeting with the treasurer in attendance the treasurer can
report the effect on the balance of club funds before the vote. The treasurer
shall prepare this section or make this report for any spending proposal
submitted to them or brought during a meeting they are present at. Proposals
cannot be considered if they spend money not currently available in the funds
they propose spending from.

\subsection{Officer Discretionary Fund}

Money from the officer discretionary fund can be spent for the benefit of the
club by consensus of the officers as they see fit, without approval from the
club membership.

\subsection{General Fund}

Money from the general fund can be spent by vote during a meeting or by approval
of more than half of the entire body of voting members. Proposals to spend money
from the general fund can be brought by any member during a meeting. Outside of
meetings, by consensus of the officers, a proposal can be brought to voting
members to seek approval by more than half of them.

\subsection{Reserve Fund}

Money from the reserve fund can only be spent as approved by two-thirds vote.
Proposals to spend money from the reserve fund can only be brought by at least
half of officers during a meeting. Other members do not have standing to propose
spending from the reserve fund, and should instead submit their proposal to the
officers.

\subsection{Grant Funds}

Each grant applied for and received has its own fund. Spending from these funds
shall be approved in the same manner as the general fund. If a grant proposal
prepared by the officers will designate received money for a specific purpose,
it is tantamount to spending the money for that purpose, and so such grant
proposals must be approved in the same way as grant fund spending before being
submitted.

\section{Constitutional Amendments}

Proposals to amend or replace the constitution can be submitted to the
membership by any member during a regular meeting if a quorum is present. The
proposed amendment will require approval from at least three quarters of voting
members to be accepted. Proposals not reaching the required level of support by
the end of the semester they are submitted in will be rejected.

The secretary shall submit any approved amendment to the Purdue University
Office of Student Activities and Organizations. The amendment will not go into
effect unless and until it is approved by this office.

\subsection{Dissolution}

The club can only be dissolved via a constitutional amendment. Proposals for
constitutional amendments that effectuate a dissolution shall provide for the
disposition of club property and funds.

\newpage

\section*{Approvals}

\vspace{4ex}

\subsection*{Members}

\begin{minipage}{0.3\textwidth}
    \begin{itemize}
        \item Member 1
        \item Member 2
        \item Member 3
        \item Member 4
    \end{itemize}
\end{minipage}
\hfill
\begin{minipage}{0.3\textwidth}
    \begin{itemize}
        \item Member 5
        \item Member 6
        \item Member 7
        \item Member 8
    \end{itemize}
\end{minipage}
\hfill
\begin{minipage}{0.3\textwidth}
    \begin{itemize}
        \item Member 9
        \item Member 10
        \item Member 11
        \item Member 12
    \end{itemize}
\end{minipage}

\vspace{6ex}

\subsection*{Officers and Officials}

\vspace{10ex}

\begin{tabular}{@{}p{3.5in}@{}}
\hrulefill \\
Robert J Sammelson \\
President of Purdue Amateur Radio Club \\
\end{tabular}

\vspace{8ex}

\noindent
\begin{tabular}{@{}p{3.5in}@{}}
\hrulefill \\
Wrigley A Starkweather \\
Treasurer of Purdue Amateur Radio Club \\
\end{tabular}

\vspace{8ex}

\noindent
\begin{tabular}{@{}p{3.5in}@{}}
\hrulefill \\
Owen M Stevens \\
Secretary of Purdue Amateur Radio Club \\
\end{tabular}

\vspace{8ex}

\noindent
\begin{tabular}{@{}p{3.5in}@{}}
\hrulefill \\
Dr. Andrew Robison \\
Advisor to Purdue Amateur Radio Club \\
\end{tabular}

\vspace{8ex}

\noindent
\begin{tabular}{@{}p{3.5in}@{}}
\hrulefill \\
Purdue Student Activities and Organizations Office \\
\end{tabular}

\end{document}
