\documentclass{article}

\usepackage[hidelinks]{hyperref}
\usepackage{microtype}

\title{\vspace*{\fill}Purdue Amateur Radio Club Constitution}
\author{Prepared By: The Officers of Purdue Amateur Radio Club}
\date{December 6, 2024\vspace*{\fill}\vspace*{\fill}}

% TODO: add provisions about what duties not specified can be done by the
% officers without a vote

\begin{document}

\maketitle

\newpage
\tableofcontents

\newpage
\setcounter{section}{-1}
\section{Preamble}

We, the members of Purdue Amateur Radio Club, declare this constitution the
legal document of Purdue Amateur Radio Club. Members of the club shall abide by
the following articles. This document shall become effective once approved by
the membership and the Student Activities and Organizations office; from that
time on, it will replace the previous constitution and all other documents of
this nature.

\section{Name}

The official name of this organization is Purdue Amateur Radio Club. The club is
also known by its Federal-Communications-Commission-assigned callsign W9YB.

\section{Purpose}

The purpose of this club is to serve as a means of maintaining an amateur radio
station for the benefit of its members, to promote goodwill among Purdue amateur
radio operators, to serve as a means of social contact for its members, and to
extend the knowledge and art of amateur radio.

\section{Membership}

\subsection{Eligibility}

Membership and participation are free from discrimination on the basis of race,
religion, color, sex, age, national origin or ancestry, genetic information,
marital status, parental status, sexual orientation, gender identity or
expression, disability, or status as a veteran. All people are eligible for
membership of the club.

Pursuant to the Purdue ``Policy Against Hazing,'' Purdue Amateur Radio Club does
not engage in hazing activities of any kind.

\subsection{Requests}

Membership requests can be introduced by any interested person during a regular
meeting or---in special circumstances and by consensus of the
officers---requests can be made to the treasurer outside a meeting. All requests
for membership shall be considered if the requisite dues have been received.
Requests made during a meeting shall be granted automatically unless an
objection to the person becoming a member has been made by at least two members,
either to the treasurer or in the meeting when the request is made. If such
objections are received, the matter of accepting the membership request will be
put to vote at the next opportunity. The officers shall only disclose who the
objectors are to non-officers during deliberations on accepting the membership
request. Requests made outside of a meeting and approved by all officers shall
be put to a vote at the next opportunity.

\subsection{Dues}

Membership dues for each available membership duration shall be prescribed by
vote during elections at the end of the prior academic year. At this time, the
club may also adopt a list of people exempt from dues for the following year.
People may also be added to the list by vote.

If the fee for a two-semester membership is increased at the end of an academic
year, members who purchased a two-semester membership during the spring term
will owe dues equal to one-half of the increase in order to be considered
current once the third regular meeting of the fall semester begins.

\subsection{Duration}

Membership terms are the current semester or the current and next semesters.
Only the fall and spring semesters are considered. The term of membership
continues until the third regular meeting of the semester after the last
semester of the membership in order to provide a period to renew.

\subsection{Types}

\subsubsection{Members}

A member is any person whose term of membership from their most recent
application has not expired and who is current in any payments due to the club.

\subsubsection{Full Members}

Full members are members who are currently Purdue University students
(undergraduate or graduate) at any Purdue University campus and currently hold a
Federal Communications Commission Amateur Radio License. During breaks between
fall and spring semesters, people who were students during the most recent fall
or spring term shall continue to be considered students.

\subsubsection{Community Members}

Community members are members who are not Purdue University students but hold a
valid Federal Communications Commission Amateur Radio License.

\subsubsection{Associate Members}

Associate members are members who are not full members or community members.

\subsection{Removal}

Members can be removed by two-thirds vote at a regular meeting with notice at
the prior regular meeting. Motions to remove members are debatable.

\section{Meetings}

\subsection{Regular Meetings}

Regular meetings are held at predetermined times during the fall and spring
semesters but never on days when classes are not in session. No more than one
regular meeting may be held on any single day. If a quorum is not present the
gathering will not qualify as a regular meeting.

\subsection{Other Meetings}

Other meetings can be called to order anytime a quorum is present to handle
business needing attention. These meetings can occur in person or via any
suitable communication mechanism that allows all attendees to hear each other
and the presiding officer to recognize who is speaking.

\subsection{Rules}

Unless otherwise provided for in the by-laws, the club shall use Robert's Rules
of Order Revised, Fourth Edition as its rules of order.

\subsection{Voting Members}

Voting members are full members who have been present at one of the prior four
regular meeting dates---whether or not a quorum is present---or the current
meeting if one is in session.

\subsection{Quorum}

A quorum shall be at least half of voting members and at least one officer. If
all officer positions are vacant, an officer shall not be required for a quorum,
and a presiding officer shall be appointed from the present voting members by
vote.

\section{Officers and Functionaries}

The club officers shall consist of the president, vice president, treasurer, and
secretary. Officers must be full members, and a person may only hold one officer
role.

The club functionaries are the advisor and the trustee. The advisor must be an
employee of Purdue University, they must not be a student of Purdue University,
and they must meet any other requirements under Purdue University student
organization rules for a club advisor. The trustee must meet all applicable
Federal Communications Commission rules for a club license trustee. One person
can hold both roles.

If an officer or functionary stops meeting their eligibility criteria their
position will be immediately vacated.

The president, treasurer, and functionaries are mandatory positions. Officers
and functionaries can vacate their position either immediately or when an
election to replace them is complete, at their option, by notifying the
officers.

\subsection{Term}

The annual officer term will begin at the last regular meeting in the spring
semester but not before the spring re-election of all officer positions is
complete. The term will last until the next annual officer term begins.

\subsection{Miscellaneous Duties}

This section specifies duties of officers in addition to duties assigned to them
by other parts of this document. Officers and functionaries are permitted to
delegate their duties as they see fit with the consent of the officers, but the
delegator is still responsible for the duty being discharged in a timely and
correct manner. Any powers or duties not otherwise delegated by the constitution
or any by-laws adopted will fall to the membership.

\subsubsection{President}

The president shall serve as chief executive officer of the club and preside
over meetings. The president shall organize members and facilitate the club's
mission, strategic plan, and activities. The president shall ensure the smooth
and collegial operation of the club. The president is responsible for the
re-registration of the club with Purdue University as needed. The president will
perform the duties of the secretary if the position is vacant.

\subsubsection{Vice President}

The vice president will assist the president in the leadership and operations of
the club. The vice president will preside in the absence of the president.

\subsubsection{Treasurer}

The treasurer is responsible for maintaining the budget, reporting club fund
balances to full members at meetings if requested, maintaining the roster of
club members, keeping accounts, depositing organization funds, and making
reimbursements or expenditures as allowed under this document in a manner
approved by the Business Office for Student Organizations. The treasurer will
preside in the absence of the president and vice president.

\subsubsection{Secretary}

The secretary is responsible for keeping meeting minutes, doing club publicity,
disseminating information about elections, announcing meetings as needed, and
tracking attendance at meetings to maintain the roster of voting members. The
secretary shall provide a mechanism for keeping meeting minutes and for
attendance to be tracked at meetings in their absence. The secretary will
preside in the absence of the president, vice president, and treasurer.

\subsubsection{Advisor}

The advisor will provide advice and suggestions regarding club operations. The
advisor is responsible for signing necessary Purdue University forms for student
organizations and overseeing that operation of the club proceeds in a manner
that complies with Purdue University rules. Advisors may not vote or hold
office as they cannot be a full member.

\subsubsection{Trustee}

The trustee is responsible for liaising with the Federal Communications
Commission including to maintain the club license(s) and is responsible for the
operation of the station and its equipment. The station license shall be in the
name of the trustee.

\subsection{Elections}

The election process for all positions will begin in the spring semester at a
time selected by the officers. The time shall be selected so that the elections
will be complete by the fifteenth of April if a quorum exists at every regular
meeting time from the start of the process until the election. The newly-elected
positions will begin their term when the next annual officer term starts.

The election process shall begin immediately if any mandatory positions are
vacant. The election process for other vacant positions may be started at any
time by unanimous vote of the officers. Officers elected to vacant positions in
these special election shall begin their term immediately, and their term will
last until the end of the current annual officer term.

\subsubsection{Election Process}

When the election process begins, it will be announced at each regular meeting
time until the election occurs. Once the announcement is made at a regular
meeting, the next regular meeting shall be the election.

Members may nominate themselves or others for any position being elected to,
including for multiple positions, but people who object to being nominated shall
not be included. Nominations and objections can be communicated to any of the
officers. Nominations are accepted from the beginning of the election process
until after the last call for nominations at the beginning of the meeting that
is the election. All people nominated for a position that do not object and that
are eligible shall be included on each ballot.

Ballots will be private and ranked choice but can be in any form approved by all
officers. Elections for functionaries will occur before officers. Elections will
be decided by the ranked pairs method, also known as the Tideman method. Ties
will be broken by the presiding officer.

The elections for officers shall be conducted in the following order: president,
treasurer, secretary, vice president. For the elections of secretary and vice
president, an option ``none'' will be included on the ballot, and if this option
is elected the positions shall be left vacant. If an officer position is left
vacant after any election, any remaining officer positions will be left vacant
and the elections will end. Once a candidate has been selected to an officer
position they will be eliminated from the ballots of the remaining officer
elections, in concert with the order elections are conducted this ensures
members running for multiple positions are voted on for the most critical
positions first.

\subsection{Removal}

Officers or functionaries can be removed, or the election process can be
initiated to replace them, by two-thirds vote at a regular meeting with notice
at the prior regular meeting. Any member can give notice or move to remove an
officer during a meeting, and such motions are subject to debate.

\section{Constitutional Amendments}

Motions to amend or replace the constitution can be introduced by any member
during a regular meeting. Once a proposal is passed, the proposed amendment will
require approval from at least three-quarters of voting members to be accepted.
Proposals not reaching the required level of support by the end of the semester
they are passed in will be rejected.

The secretary shall submit any approved amendment to the Purdue University
Office of Student Activities and Organizations. The amendment will not go into
effect unless and until it is approved by this office.

\subsection{Dissolution}

The club can only be dissolved via a constitutional amendment. Proposals for
constitutional amendments that effectuate a dissolution shall provide for the
disposition of club property and funds.

\section{By-laws}

By-laws can be adopted, if none exist, or amended at a regular meeting---so long
as they do not conflict with the constitution---by three-quarters vote, so long
as notice was given at the previous regular meeting. The by-laws may specify the
rules of order for any motion mentioned in this constitution. Any member can
move to adopt or amend by-laws.

% \newpage

% \section*{Approvals}

% \vspace{4ex}

% \subsection*{Members}

% \begin{minipage}{0.3\textwidth}
%     \begin{itemize}
%         \item Member 1
%         \item Member 2
%         \item Member 3
%         \item Member 4
%     \end{itemize}
% \end{minipage}
% \hfill
% \begin{minipage}{0.3\textwidth}
%     \begin{itemize}
%         \item Member 5
%         \item Member 6
%         \item Member 7
%         \item Member 8
%     \end{itemize}
% \end{minipage}
% \hfill
% \begin{minipage}{0.3\textwidth}
%     \begin{itemize}
%         \item Member 9
%         \item Member 10
%         \item Member 11
%         \item Member 12
%     \end{itemize}
% \end{minipage}

% \vspace{6ex}

% \subsection*{Officers and Officials}

% \vspace{10ex}

% \begin{tabular}{@{}p{3.5in}@{}}
% \hrulefill \\
% Robert J Sammelson \\
% President of Purdue Amateur Radio Club \\
% \end{tabular}

% \vspace{8ex}

% \noindent
% \begin{tabular}{@{}p{3.5in}@{}}
% \hrulefill \\
% Wrigley A Starkweather \\
% Treasurer of Purdue Amateur Radio Club \\
% \end{tabular}

% \vspace{8ex}

% \noindent
% \begin{tabular}{@{}p{3.5in}@{}}
% \hrulefill \\
% Owen M Stevens \\
% Secretary of Purdue Amateur Radio Club \\
% \end{tabular}

% \vspace{8ex}

% \noindent
% \begin{tabular}{@{}p{3.5in}@{}}
% \hrulefill \\
% Dr. Andrew Robison \\
% Advisor to Purdue Amateur Radio Club \\
% \end{tabular}

% \vspace{8ex}

% \noindent
% \begin{tabular}{@{}p{3.5in}@{}}
% \hrulefill \\
% Purdue Student Activities and Organizations Office \\
% \end{tabular}

\end{document}
